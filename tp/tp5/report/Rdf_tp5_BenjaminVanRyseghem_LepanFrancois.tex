 \documentclass[a4paper,10pt]{article}
\input{macro.tex}

\title{Rdf - TP5 : Classement de caract�re manuscrits}
\author{Fran�ois \bsc{Lepan} - Benjamin \bsc{Van Ryseghem}}

\begin{document}
\maketitle 

\section{Visualiser les Donnees}

\begin{verbatimtab}
y = matrix(train0(1,:),28,28)';
Matplot(y);
\end{verbatimtab}

Afin de de visualiser les autres chiffres manuscrits il suffit de changer le nom de train0 en train$N$ o� $N$ est le chiffre � visualiser.

\section{L'algorithme de plus proche voisin ou kNN}

\subsection{Est-ce que les r�sultats changent si $\displaystyle d^2(x, y)=\Sigma^n_{i=1}(xi - yi)^2$ est utilis�e comme distance?}

Non car tout ce que l'on fait c'est comparer ces distances ($\geq 0$) donc leurs valeurs �lev� au carr� ne changeront rien lors de la comparaison.

Ex: si a = 2 et b = 3

$a < b \Rightarrow a^2 < b^2$

$2 < 3 \Rightarrow 4 < 9$

\subsection{Quelles sont les erreurs moyennes de classification pour k = 1, 3, 5, 10?}

Pour k = 1 on a une erreur moyenne de 0.6885923. 

�tant donn� que la plupart des gens on 0.1 je me dit qu'il doit y avoir une erreur dans notre code mais je n'arrive pas � savoir o�.

\subsection{Cette approche est-elle supervis�e ?}

Oui cette approche est supervis�e, car il faut l'alimenter de donn�es d'entrainement afin de savoir comment classifier le nouvel �l�ment entrant.

\section{L�algorithme Na�ve Bayes}

\subsection{Quelle est l�hypoth�se principale dans cette approche?}

$P(X = (x1,�,xm~|~Ci)) = \prod_{j=1}^m P(xj | Ci)$ \\
\\
 Propri�t�: L'ind�pendance conditionnelle des attributs donc on peut faire le produit de ses �l�ments.
 
\subsection{Calcul des erreurs de classification}

Pour les erreurs de classifications nous avons chang�es valeurs pour n=200 et n =1000 mais nous avons toujours le m�me r�sultat 0.902.

Nous avons cherch� mais nous n'avons pas trouv� la source de l'erreur.

\subsection{Cette approche est-elle supervis�e ?}

Oui cette approche est supervis�e, car elle utilise des donn�es d'entrainement tout comme kNN.

\subsection{Cette approche est-elle probabiliste ?}

Oui car elle est bas�e sur la fr�quence de chaque valeur (ici 0 $\rightarrow$ 255) quelque soit le point d�entrainement et quel que soit l�emplacement de cette valeur dans le point.

\end{document}