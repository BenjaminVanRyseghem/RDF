 \documentclass[a4paper,10pt]{article}
\input{/Users/WannaGetHigh/workspace/latex/macros.tex}

\title{TP - RdF: Arbres de d�cision}
\author{Fran�ois \bsc{Lepan}}

\begin{document}
\maketitle

\section{Question de bon sens}
\subsection{Que vaut N ?}

N vaut $2^{4+1}$ car pour quatre propositions on a toujours que deux possibilit� et le +1 pour arriv� � la solution final.

\subsection{Expliquer le raisonnement de B et calculer la quantit� d�information que lui a donn�e A par sa r�ponse}

???????????????????\\
???????????????????\\

\section{Variante du jeu du pendu}

\subsection{But du jeu}

\begin{paragraph}{Si le programme est s�r de trouver la solution en au plus p questions, quelle est la relation entre n et p ? En d�duire une valeur num�rique minimale pour p.} ~ \\

??????????????? \\
??????????????? \\
??????????????? 

\end{paragraph}

\subsection{Que fait ce code ?}

\begin{verbatimtab}

mat = zeros(35,n);

for i = 1 : n
	c = str2code(noms(i));
	mat(c,i) = 1;
end
\end{verbatimtab}

Ce code stock dans \emph{mat}, en colonnes, les lettres du mot qui sont pr�sentes (35 car "a" == 10).
Pour cela on met 1 si la lettre est pr�sente (si elle est la deux fois on ne le saura pas).

\subsection{Quelle lettre est la plus repr�sent�e ?}

\begin{Verbatim}[commandchars=\\\{\}]

\codeBlue{// retourne une matrice de 283 colonnes et 35 lignes contenant}
\codeBlue{// pour chaque colonne un 1 ou un 0 si il contient la lettre correspondant �}
\codeBlue{// l'indice (a � pour indice 10)}
\codeRed{function} mat = presentLettersIn(noms)

    n = size(noms,1);
    
    mat = zeros(35,n);
    
    for i = 1 : n
        c = str2code(noms(i));
        mat(c,i) = 1;
    end

\codeRed{endfunction}


\codeRed{function} h = getNumberOfLetters(noms)
    
    \codeBlue{// on cr�er une colonne de 283 afin de faire une multiplication}
    \codeBlue{// entre cette colonne et la matrice de 1 et 0 retourn�e par presentLettersIn}
    \codeBlue{// afin de r�cup�rer une seul colonne de 35 contenant pour chaque element}
    \codeBlue{// la somme des �l�ments de sa ligne.}
    one = ones(size(noms,1),1);
    
    m = presentLettersIn(noms)
    
    h = m * one;

\codeRed{endfunction}



\codeRed{function} letter = mostPresentLettersIn(noms)
    
    h = getNumberOfLetters(noms);
    
    \codeBlue{// On r�cup�re l'indice de la lettre la plus pr�sente}
    [value, mostPresentLetterCode] = max(h);

    \codeBlue{// on transforme l'indice en lettre}
    letter = code2str(mostPresentLetterCode);

\codeRed{endfunction}

\codeBlue{// Chargement de la base de noms d'animaux}
exec("rdfAnimaux.txt");

\codeBlue{// on r�cup�re la lettre la plus pr�sente dans tous les mots}
letter = mostPresentLettersIn(noms)

\end{Verbatim}

Apr�s ex�cution du code on obtient la lettre e.

\subsection{Entropie}

\begin{Verbatim}[commandchars=\\\{\}]
\codeRed{function} resultat = entropie(noms, n)
    
    m = getNumberOfLetters(noms);
    
    resultat = - log((m./n).^(m./n)) - log((1 - m./n).^(1 - m./n))

\codeRed{endfunction}
\end{Verbatim}

Voici la fonction qui permet de calcul� l'entropie

\subsection{Partage}
?????????????????? \\
?????????????????? \\
??????????????????


\subsection{Que fait ce code ?}

\begin{verbatimtab}
I=([1:n]>0);

[A,B,i]=partage(I);code2str(i),sum(A),sum(B)
[C,D,j]=partage(A);code2str(j),sum(C),sum(D)
[E,F,k]=partage(B);code2str(k),sum(E),sum(F)
\end{verbatimtab}

Il affiche la r�partition des mots dans l'arbre. Pour le premier partage on � 140 mots dans A et 143 dans B. Pour le partage de A on obtient deux ensembles C contenant 71 mots et D 69 mots. Et enfin pour le partage de B on obtient deux ensembles C contenant 70 mots et D 73 mots.

Si on continu on aura pour chaque noeud le nombre de fils qu'il poss�de, pour chacun de ses fils le nombre de fils, etc.

\subsection{L'arbre}

\begin{paragraph}{Profondeur}~\\

Afin de conna�tre la profondeur maximale de l'arbre il suffit de rajouter ces lignes en bleu � la fonction \emph{arbre}:

\begin{Verbatim}[commandchars=\\\{\}]
\codeRed{function} \codeBlue{maxim} = arbre(I,str)
    if (sum(I)>1) then
        
        [A,B,i] = partage(I);
        \codeBlue{max1 =} arbre(A,str+'| ');
        printf("\%s\%c (\%i,\%i)",str,code2str(i),sum(A),sum(B));
        \codeBlue{max2 =} arbre(B,str+'| ');
        
        \codeBlue{maxim = 1 + max(max1,max2);}
        
    else
        printf("\%s -> \%s",str,noms(I));
        \codeBlue{maxim = 0;}
    end

\codeRed{endfunction}

\end{Verbatim}

Le r�sultat est 10.

\end{paragraph}
~\\
~\\

\begin{paragraph}{Mots d�favorables}~\\

Afin d'afficher les mots les plus d�favorables il faut rajouter ces lignes en bleu � la fonction \emph{arbre}: 

\begin{Verbatim}[commandchars=\\\{\}]
\codeRed{function} maxim = arbre(I,str\codeBlue{,currentDeepth,maxDeepth})

    if (sum(I)>1) then

        \codeBlue{currentDeepth = currentDeepth +1;}

        [A,B,i] = partage(I);
        
        max1 = arbre(A,str+'| ',currentDeepth,maxDeepth);
        
        printf("\%s\%c (\%i,\%i)",str,code2str(i),sum(A),sum(B));
        
        max2 = arbre(B,str+'| ',currentDeepth,maxDeepth);
        
        maxim = 1 + max(max1,max2);

    else

        \codeBlue{if currentDeepth == maxDeepth then}
            \codeBlue{global deepestElements;}
            \codeBlue{deepestElements = [deepestElements noms(I)];}
        \codeBlue{end}
        
        printf("\%s -> \%s",str,noms(I));
        maxim = 0;
    end

\codeRed{endfunction}
\end{Verbatim}

Ainsi qu'une variable globale: \emph{global deepestElements;}

\begin{Verbatim}[commandchars=\\\{\}]

\codeBlue{global deepestElements;}
// pour le vider � chaque ex�cution
\codeBlue{deepestElements = [];}

maximum = arbre(I,' '\codeBlue{,0,10})

\end{Verbatim}

L'ex�cution du code pr�c�dent nous fournis les animaux suivant: roussette,  tortue, hareng, panthere, merlan 
et varan.
\end{paragraph}
~\\
~\\

\begin{paragraph}{Moyenne}~\\

Afin d'avoir en plus le nombre moyen n�cesssaire pour trouver la solution en suivant cet arbre il faut ajouter les lignes en bleu � la fonction \emph{arbre}:  

\begin{Verbatim}[commandchars=\\\{\}]
\codeRed{function} \codeBlue{[nbQuestion,} maxim\codeBlue{]} = arbre(I,str,currentDeepth,maxDeepth)

    if (sum(I)>1) then

        currentDeepth = currentDeepth +1;

        [A,B,i] = partage(I);
        
        \codeBlue{[nbQuestion1,} max1\codeBlue{]} = arbre(A,str+'| ',currentDeepth,maxDeepth);
        
        printf("\%s\%c (\%i,\%i)",str,code2str(i),sum(A),sum(B));
        
       \codeBlue{[nbQuestion2,} max2\codeBlue{]} = arbre(B,str+'| ',currentDeepth,maxDeepth);
        
        maxim = 1 + max(max1,max2);
        \codeBlue{nbQuestion = nbQuestion1 + nbQuestion2;}

    else

        if currentDeepth == maxDeepth then
            global deepestElements;
            deepestElements = [deepestElements noms(I)];
        end
        
        \codeBlue{nbQuestion = currentDeepth;}
        
        printf("\%s -> \%s",str,noms(I));
        maxim = 0;
    end

\codeRed{endfunction}
\end{Verbatim}

Ainsi que diviser le nombre totale de question par le nombre d'animaux.

\begin{Verbatim}[commandchars=\\\{\}]
\codeBlue{[nbQuestion,} maximum\codeBlue{]} = arbre(I,' ',0,10);

\codeBlue{moyenne = nbQuestion / n}
\end{Verbatim}

Comparer a P !!!!!!!!!!!!!!!!!!!!!!!\\
?????????????????????\\
?????????????????????\\

\end{paragraph}


\end{document}




















