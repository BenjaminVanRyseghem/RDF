 \documentclass[a4paper,10pt]{article}
\input{/Users/WannaGetHigh/workspace/latex/macros.tex}

\title{TP2 : Codage d'un contour}
\author{Benjamin \bsc{Van Ryseghem} Fran�ois \bsc{Lepan}}

\begin{document}
\maketitle

\section{Codage d'un contour}

\subsection{Cont}
Les points du contour sont stock� dans la variable sous forme d'une liste de nombre complexe.

\subsection{}
\begin{Verbatim}
nom1 = "cercle-80.txt";
cont1 = rdfChargeFichierContour (nom1);

cont2 = cont1(1:4:$);
cont3 = cont1(1:8:$);
rdfAfficheContour(cont1, 2, "r");
rdfAfficheContour(cont2, 2, "g");
rdfAfficheContour(cont3, 2, "b");
\end{Verbatim}

\section{Descripteur de Fourier}

\subsection{Pourquoi la fonction rdfDescFourier �limine t'elle parfois un point de la liste des points du contour?}

Car si il y a un nombre paire de point alors on ne peut pas prendre le point correspondant au milieu de la liste de point.

\subsection{}

\begin{Verbatim}
descFour = rdfDescFourier(cont);
rdfAfficheContour (rdfInverseDescFourier(descFour), 1, "k");
\end{Verbatim}

\subsection{Quel est l'indice de tableau correspondant au descripteur Z0 ?}
Il se situe � la moiti� de la taille du tableau.

\subsection{Expliquer l'utilit� de la fonction rdfValeurDescFourier}
Cette fonction sert � afficher la valeur d'un point de la liste des descripteurs de Fourier 
 
 \subsection{A quoi correspond le descripteur de Fourier Z0 d'une forme d�crite par son contour?}
 C'est le barycentre de la forme.
 
 On observe que si on le modifie ................... ???




\end{document}