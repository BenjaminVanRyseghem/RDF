\documentclass[a4paper,10pt]{article}
\input{/Users/WannaGetHigh/workspace/latex/macros.tex}

\title{R�gression Logistique}

\author{Fran�ois \bsc{Lepan} - Benjamin \bsc{Van Ryseghem}}

\begin{document}
\maketitle

Voici les �tapes principales de l'algorithme de r�gression logistique.

Ici on utilise comme donn�es d'entrainement les \emph{m} images (x) avec leurs \emph{m} �tiquettes (y). 

\begin{paragraph}{Entrainement}~ \\
\\
Tout d'abord il faut entra�ner un classificateur de r�gression logistique 
$\displaystyle h^{(i)}_\theta = \frac{1}{1+e^{\theta^{T_x}}}$ 
pour chaque classe i. 

Pour l'entrainer il faut trouver le vecteur $\theta$ qui s�parera la classe i des autres classes. 

Ce vecteur est trouv� par la relation suivante:  \\
\\
$\displaystyle \theta^* = \min \limits_\theta  J(\theta)$ o� \\
\\
$\displaystyle J(\theta) = - \frac{1}{m} \sum \limits_{i=1}^m y^{(i)} \log(h_\theta (x^{(i)})) - (1 - y^{(i)}) \log(1 - h_\theta (x^{(i)}))$ \\
\\

\end{paragraph} 

Ensuite lorsqu'on � fini de faire les entrainements de chaque classificateur il reste � faire la pr�diction de la classe de la nouvelle image x.

\begin{paragraph}{Pr�diction}~ \\
\\
Afin de pr�dire la classe d'une nouvelle image x il faut choisir sont �tiquette y tel que : \\
\\
$\displaystyle y = \max \limits_{i = 1...l} h^{(i)}_\theta$ i.e on choisi le classificateur le plus confiant.\\
\\
Puis on r�p�te le processus pour chaque nouvelle image.
\end{paragraph}

\end{document}