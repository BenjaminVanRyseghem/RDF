\documentclass[a4paper,10pt]{article}
\input{/Users/WannaGetHigh/workspace/latex/macros.tex}

\title{RDF - TP 1}
\author{Fran�ois \bsc{Lepan}\\Benjamin \bsc{Van Ryseghem}}

\begin{document}
\maketitle

\section{Code Scilab}
\subsection{Analysez le code contenu dans ce fichier et expliquez comment les doubles sommes n�cessaires au calcul des moments g�om�triques sont implant�es. Quel est l'int�r�t de cette technique?}
??????

\section{Moments d'une forme}
\subsection{Calcule des valeurs propres et les vecteurs propres de la matrice d'inertie des 4 rectangles}
\subsubsection{ Rectangle Horizontal}
\paragraph{Valeurs Propres}
\[ \left( 80\  1360 \right)\]

\paragraph{Vecteurs Propres}
\[
   \left (
   \begin{array}{cc}
      0 & 1  \\
      1 & 0 \\
   \end{array}
   \right )
\]

\paragraph{Axe Principale}
\[
   \left (
   \begin{array}{c}
      1  \\
      0\\
   \end{array}
   \right )
\]


\paragraph{Moments Principaux}
\[ \left( 80 \ 1360 \right)\]

\subsubsection{ Rectangle Vertical}
\paragraph{Valeurs Propres}
\[ \left( 80\  1360 \right)\]

\paragraph{Vecteurs Propres}
\[
   \left (
   \begin{array}{cc}
      1 & 0  \\
      0 & 1 \\
   \end{array}
   \right )
\]

\paragraph{Axe Principale}
\[
   \left (
   \begin{array}{c}
      0  \\
      1\\
   \end{array}
   \right )
\]


\paragraph{ Moments Principaux}
\[ \left( 80 \ 1360 \right)\]

 
 \subsubsection{ Rectangle Diagonal}
\paragraph{Valeurs Propres}
\[ \left( 59 \  1298 \right)\]

\paragraph{Vecteurs Propres}
\[
   \left (
   \begin{array}{cc}
      - 0.7071068 & - 0.7071068  \\
      - 0.7071068 &  0.7071068 \\
   \end{array}
   \right )
\]

\paragraph{Axe Principale}
\[
   \left (
   \begin{array}{c}
      - 0.7071068  \\
       0.7071068\\
   \end{array}
   \right )
\]


\paragraph{ Moments Principaux}
\[ \left( 1298 \ 59 \right)\]

 \subsubsection{ Rectangle Diagonal Liss�}
\paragraph{Valeurs Propres}
\[ \left( 99.673034 \  1393.7427 \right)\]

\paragraph{Vecteurs Propres}
\[
   \left (
   \begin{array}{cc}
      - 0.7080350 & - 0.7061774  \\
      - 0.7061774 &  0.7080350 \\
   \end{array}
   \right )
\]

\paragraph{Axe Principale}
\[
   \left (
   \begin{array}{c}
      - 0.7061774  \\
       0.7080350\\
   \end{array}
   \right )
\]


\paragraph{ Moments Principaux}
\[ \left( 99.673034 \ 1393.7427 \right)\]
 
\subsection{Quelle est la diff�rence entre les deux images d'un rectangle diagonal?}
 ??????
\subsection{Comment cela influence t'il le calcul des moments?}
 ??????
\subsection{Calcule des moments principaux d'inertie des diff�rents carr�s (6, 10, 30deg, 45deg)}
m  =
 
    105.  
    105.  
-->m1 = inertiaMatrix(image1);
-->m1 = momentums(m1)
 m1  =
 
    825.  
    825.  
-->m2 = inertiaMatrix(image2);
-->m2 = momentums(m2)
 m2  =
 
    842.42024  
    843.28148  
-->m3 = inertiaMatrix(image3);
-->m3 = momentums(m3)
 m3  =
 
    841.51713  
    838.53593  
-->m4 = inertiaMatrix(image4);
-->m4 = momentums(m4)
 m4  =
 
    13300.  
    13300.  
-->m5 = inertiaMatrix(image5);
-->m5 = momentums(m5)
 m5  =
 
    396.02318  
    420.81221  
    
+ Conclure sur la possibilit� d'utiliser ces moments comme atributs de forme.
 
 \section{Moments normalis�s}
 
 \subsection{Fonction rdfMomentCentreNormalise  -> etha }
 
 
 \subsection{}
-->m = inertiaMatrixCentered(image);
-->m = momentums(m)
 m  =
 
    0.0810185  
    0.0810185  
-->m1 = inertiaMatrixCentered(image1);
-->m1 = momentums(m1)
 m1  =
 
    0.0825  
    0.0825  
-->m2 = inertiaMatrixCentered(image2);
-->m2 = momentums(m2)
 m2  =
 
    0.0840244  
    0.0841103  
-->m3 = inertiaMatrixCentered(image3);
-->m3 = momentums(m3)
 m3  =
 
    0.0854334  
    0.0851307  
-->m4 = inertiaMatrixCentered(image4);
-->m4 = momentums(m4)
 m4  =
 
    0.083125  
    0.083125  
-->m5 = inertiaMatrixCentered(image5);
-->m5 = momentums(m5)
 m5  =
 
    0.3320313  
    0.0195313  
-->m6 = inertiaMatrixCentered(image6);
-->m6 = momentums(m6)
 m6  =
 
    0.0195313  
    0.3320313  
-->m7 = inertiaMatrixCentered(image7);
-->m7 = momentums(m7)
 m7  =
 
    0.3858502  
    0.0175386  
-->m8 = inertiaMatrixCentered(image8);
-->m8 = momentums(m8)
 m8  =
 
    0.0239599  
    0.3350345  
    
    \subsection{calcule des moments principaux d'inertie en diagonalisant la matrice d'inertie calcul�e � partir des moments centr�s normalis�s plut�t qu'� partir des moments centr�s}
    ???????
    
    \section{Moments invariants}
    
    \subsection{Calcule des attributs des formes contenues dans les images d'une m�me forme pour diff�rentes orientations et diff�rentes �chelles (les carr�s)}
    -->m = Hu5(image)
 m  =
 
    0.  
-->m1 = Hu5(image1)
 m1  =
 
    0.  
-->m2 = Hu5(image2)
 m2  =
 
    1.378D-14  
-->m3 = Hu5(image3) 
 m3  =
 
    6.501D-15  
-->m4 = Hu5(image4)
 m4  =
 
    0.  
%\signature

\end{document}