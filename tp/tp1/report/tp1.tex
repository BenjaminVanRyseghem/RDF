\documentclass[a4paper,10pt]{article}
\input{../../../../../LaTeX/macro.tex}

\title{RDF - TP 1}
\author{Fran�ois \bsc{Lepan}\\Benjamin \bsc{Van Ryseghem}}

\begin{document}
\maketitle

\section{Code Scilab}
\subsection{Analysez le code contenu dans ce fichier et expliquez comment les doubles sommes n�cessaires au calcul des moments g�om�triques sont implant�es. Quel est l'int�r�t de cette technique?}

Les doubles sommes sont implant�es en utilisant les propri�t�s des calculs matriciels afin d'�viter les boucles d'it�rations.
L'int�r�t de cette technique est qu'�tant donn� que les calculs matriciels sont tr�s optimis�s, il y a un gain de performance (et de lisibilit�).

\section{Moments d'une forme}
\subsection{Calcule des valeurs propres et les vecteurs propres de la matrice d'inertie des 4 rectangles}
\subsubsection{ Rectangle Horizontal}
\paragraph{Valeurs Propres}
\[ \left( 80\  1360 \right)\]

\paragraph{Vecteurs Propres}
\[
   \left (
   \begin{array}{cc}
      0 & 1  \\
      1 & 0 \\
   \end{array}
   \right )
\]

\paragraph{Axe Principale}
\[
   \left (
   \begin{array}{c}
      1  \\
      0\\
   \end{array}
   \right )
\]


\paragraph{Moments Principaux}
\[ \left( 80 \ 1360 \right)\]

\subsubsection{ Rectangle Vertical}
\paragraph{Valeurs Propres}
\[ \left( 80\  1360 \right)\]

\paragraph{Vecteurs Propres}
\[
   \left (
   \begin{array}{cc}
      1 & 0  \\
      0 & 1 \\
   \end{array}
   \right )
\]

\paragraph{Axe Principale}
\[
   \left (
   \begin{array}{c}
      0  \\
      1\\
   \end{array}
   \right )
\]


\paragraph{ Moments Principaux}
\[ \left( 80 \ 1360 \right)\]

 
 \subsubsection{ Rectangle Diagonal}
\paragraph{Valeurs Propres}
\[ \left( 59 \  1298 \right)\]

\paragraph{Vecteurs Propres}
\[
   \left (
   \begin{array}{cc}
      - 0.7071068 & - 0.7071068  \\
      - 0.7071068 &  0.7071068 \\
   \end{array}
   \right )
\]

\paragraph{Axe Principale}
\[
   \left (
   \begin{array}{c}
      - 0.7071068  \\
       0.7071068\\
   \end{array}
   \right )
\]


\paragraph{ Moments Principaux}
\[ \left( 1298 \ 59 \right)\]

 \subsubsection{ Rectangle Diagonal Liss�}
\paragraph{Valeurs Propres}
\[ \left( 99.673034 \  1393.7427 \right)\]

\paragraph{Vecteurs Propres}
\[
   \left (
   \begin{array}{cc}
      - 0.7080350 & - 0.7061774  \\
      - 0.7061774 &  0.7080350 \\
   \end{array}
   \right )
\]

\paragraph{Axe Principale}
\[
   \left (
   \begin{array}{c}
      - 0.7061774  \\
       0.7080350\\
   \end{array}
   \right )
\]


\paragraph{ Moments Principaux}
\[ \left( 99.673034 \ 1393.7427 \right)\]
 
\subsection{Quelle est la diff�rence entre les deux images d'un rectangle diagonal?}

Entre les deux rectangles, m�me si l'axe principal reste a peu de choses pr�s le m�me, l'orientation est diff�rente.

\subsection{Comment cela influence t'il le calcul des moments?}
De ce fait, les moments sont eux aussi invers�s. � cause de cela, deux figures proches ont des moments principaux vraiment diff�rents.


\subsection{Calcule des moments principaux d'inertie des diff�rents carr�s (6, 10, 30deg, 45deg)}
m  =
 
    105.  
    105.  
-->m1 = inertiaMatrix(image1);
-->m1 = momentums(m1)
 m1  =
 
    825.  
    825.  
-->m2 = inertiaMatrix(image2);
-->m2 = momentums(m2)
 m2  =
 
    842.42024  
    843.28148  
-->m3 = inertiaMatrix(image3);
-->m3 = momentums(m3)
 m3  =
 
    841.51713  
    838.53593  
-->m4 = inertiaMatrix(image4);
-->m4 = momentums(m4)
 m4  =
 
    13300.  
    13300.  
-->m5 = inertiaMatrix(image5);
-->m5 = momentums(m5)
 m5  =
 
    396.02318  
    420.81221  
    
+ Conclure sur la possibilit� d'utiliser ces moments comme atributs de forme.
 
 \section{Moments normalis�s}
 
 \subsection{Fonction rdfMomentCentreNormalise : $\eta$ }
 
 
\begin{verbatimtab}
m = inertiaMatrixCentered(image);
	0.0810185
momentums(m)
	0.0810185  
	
inertiaMatrixCentered(image1);
	0.0825
momentums(m1)
	0.0825  
	
inertiaMatrixCentered(image2);
	0.0840244
momentums(m2)
	0.0841103  
	
inertiaMatrixCentered(image3);
	0.0854334
momentums(m3)
	0.0851307  
	
inertiaMatrixCentered(image4);
	0.083125
momentums(m4)
	0.083125  
	
inertiaMatrixCentered(image5);
	0.3320313
momentums(m5)
	0.0195313  
	
inertiaMatrixCentered(image6);
	0.0195313
momentums(m6)
	0.3320313  
	
inertiaMatrixCentered(image7);
	0.3858502
momentums(m7)
	0.0175386  
	
inertiaMatrixCentered(image8);
	0.0239599
momentums(m8)
	0.3350345  
	
\end{verbatimtab}
    
    \subsection{Calcul des moments principaux d'inertie en diagonalisant la matrice d'inertie calcul�e � partir des moments centr�s normalis�s plut�t qu'� partir des moments centr�s}
    ???????
    
    \section{Moments invariants}
    
    \subsection{Calcule des attributs des formes contenues dans les images d'une m�me forme pour diff�rentes orientations et diff�rentes �chelles (les carr�s)}
\begin{verbatimtab}   
Hu5(image)
	0.  
	
Hu5(image1)
 	0.  

Hu5(image2)
 	1.378D-14.
	
Hu5(image3) 
	6.501D-15.
	
Hu5(image4)
	0.  
\end{verbatimtab}
\signature
\end{document}