 \documentclass[a4paper,10pt]{article}
\input{/Users/WannaGetHigh/workspace/latex/macros.tex}

\title{Td 1 : attributs de formes}
\author{Fran�ois \bsc{Lepan}}

\begin{document}
\maketitle

\section{Moments d'une forme}
Calculer :
\subsection{Les moments  M00, M10 et M01 de cette forme}

\[ M_{ij} = Ens_x Ens_y x^i y^j I(x,y) \]
\begin{itemize}
\item \[ M_{00} = 128 * 20 = 2560 \]

\item \[ M_{10} =  Ens_{x=0}^9 ( x * Ens_{y=0}^7 I(x,y) )\]
	\[ M_{10} = 2* (2*128) + 3* (4 * 128) + 4 * (4 *128) + 5 * (4 *128) + 6 * (4 *128) + 7 * (2 *128) = 11520 \]

\item \[ M_{01} =  Ens_{x=0}^9 ( x * Ens_{y=0}^7 I(x,y) )\]
	 \[ M_{01} = 2*( 4*128) + 3* (6 * 128) + 4 * (6 *128) + (5 * 4) *128 =   \]	 
\end{itemize}

\subsection{Les moments  M00, M10 et M01 de cette forme}
\[ (x,y) = (M_{10} / M_{00} , M_{01} / M_{00} )  = (4.5 , 3.5) \] 
\[ \mu_10 = Ens_x Ens_y (x-\overline{x})I \]


section{}

pour la signature polaire faire :
une courbe avec pour y : r(TETA) / x: TETA (jusque 2pi) puis il faut calculer cette courbe.
pour le triangle en partant du point tt a droite on va avoir  3 U.

\end{document}

			