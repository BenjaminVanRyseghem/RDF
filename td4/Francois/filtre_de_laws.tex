 \documentclass[a4paper,10pt]{article}
\input{/Users/WannaGetHigh/workspace/latex/macros.tex}

\title{TD : Les filtres de Laws}
\author{Fran�ois \bsc{Lepan}}

\begin{document}
\maketitle

\section{Filtres de LAWS}
\subsection{calculer les 5 images r�sultant du filtrage d'une image ligne contenant que des pixels blanc ou noire}

Pour un pixel x de l'image et le filtre L5 (1, 4, 6, 4, 1) on a:
$Res(x) = 1 *val(x-2) + 4 *val(x-1) + 6 *val(x) + 4 *val(x+1) + 1 *val(x+2)$

Si la somme des coefficient est nulle alors le r�sultat sera null.
Ici on ne consid�re pas les bords

\begin{paragraph} {Filtre L5: 1, 4, 6, 4, 1}~ \\
Image noire: -1,-1,-1,-1,-1  $\rightarrow$ res = -16 \\
Image blanche: +1,+1,+1,+1,+1 $\rightarrow$  res = 16
\end{paragraph}

\begin{paragraph} {Filtre E5: -1, -2, 0, 2, 1}~ \\
Image noire: -1,-1,-1,-1,-1  $\rightarrow$ res = 0 \\
Image blanche: +1,+1,+1,+1,+1 $\rightarrow$  res = 0
\end{paragraph}

\begin{paragraph} {Filtre S5: -1, 0, 2, 0, -1}~ \\
Image noire: -1,-1,-1,-1,-1  $\rightarrow$ res = 0 \\
Image blanche: +1,+1,+1,+1,+1 $\rightarrow$  res = 0
\end{paragraph}

\begin{paragraph} {Filtre W5: -1, 2, 0, -2, 1}~ \\
Image noire: -1,-1,-1,-1,-1  $\rightarrow$ res = 0 \\
Image blanche: +1,+1,+1,+1,+1 $\rightarrow$  res = 0
\end{paragraph}

\begin{paragraph} {Filtre R5: 1, -4, 6, -4, 1}~ \\
Image noire: -1,-1,-1,-1,-1  $\rightarrow$ res = 0 \\
Image blanche: +1,+1,+1,+1,+1 $\rightarrow$  res = 0
\end{paragraph}


\subsection{calculer les 5 images r�sultant du filtrage d'une image ligne contenant une alternance de pixels blanc et noire}

\begin{paragraph} {Filtre L5: 1, 4, 6, 4, 1}~ \\
Image depart: -1, +1, -1, +1, -1  $\rightarrow$ res = 0 \\
Image depart +1, -1, +1, -1, +1 $\rightarrow$  res = 0
\end{paragraph}

\begin{paragraph} {Filtre E5: -1, -2, 0, 2, 1}~ \\
Image depart: -1, +1, -1, +1, -1  $\rightarrow$ res = 0 \\
Image depart +1, -1, +1, -1, +1 $\rightarrow$  res = 0
\end{paragraph}

\begin{paragraph} {Filtre S5: -1, 0, 2, 0, -1}~ \\
Image depart: -1, +1, -1, +1, -1  $\rightarrow$ res = 0 \\
Image depart +1, -1, +1, -1, +1 $\rightarrow$  res = 0
\end{paragraph}

\begin{paragraph} {Filtre W5: -1, 2, 0, -2, 1}~ \\
Image depart: -1, +1, -1, +1, -1  $\rightarrow$ res = 0 \\
Image depart +1, -1, +1, -1, +1 $\rightarrow$  res = 0
\end{paragraph}

\begin{paragraph} {Filtre R5: 1, -4, 6, -4, 1}~ \\
Image depart: -1, +1, -1, +1, -1  $\rightarrow$ res = -16 \\
Image depart +1, -1, +1, -1, +1 $\rightarrow$  res = 16
\end{paragraph}

\subsection{combien d'alternances +/- trouve-t-on dans la s�rie de co�fficients d�finissant chaque filtre ?}

Filtre L5: 0\\
Filtre E5: 1\\
Filtre S5: 2\\
Filtre W5: 3\\
Filtre R5: 4 

\subsection{Pour chaque filtre de Laws, peut-on trouver une texture, c'est-�-dire un arrangement p�riodique de valeurs binaires, pour lequel la valeur absolue de la r�ponse est le plus souvent maximale ?}

\begin{paragraph} {Filtre L5: 1, 4, 6, 4, 1}~ \\
Valeur maximale = 1, 1, 1, 1, 1, 1, 1, 1 \\
Valeur maximale res = 16, 16, 16, 16, 16, 16,...
Changement de signe pour une p�riodicit� de 8 : 0
\end{paragraph}

\begin{paragraph} {Filtre E5: -1, -2, 0, 2, 1}~ \\

\begin{tabular}{|c|c|c|c|c|c|c|c|c|c|c|c|}
 \hline
& & & & & & & & & & & resultat (doit toujours etre max) \\
 \hline
texture &-1 & -1 & -1 & 1 & 1 & 1 & 1 & -1 & -1 & -1 &  (valeurs rajout�es pr avoir une val max)\\
\hline
arrangement & 1 & 2 & 0 & 2 & 1 & & &  &&& 6 \\
\hline
arrangement &  & 1 & 2 & 0 & 2 & 1 &  &  &&&  6\\
\hline
arrangement &  &  & 1 & -2 & 0 & 2 & 1 &  & && 2\\
\hline
arrangement &  &  &  & -1 & -2 & 0 & 2 & -1 & && -2\\
\hline
arrangement &  &  &  &  & -1 & -2 & 0 & -2 &-1 && -6\\
\hline
arrangement &  &  &  &  &  & -1 & -2 & 0 &-2 &-1& -6\\
\hline
\end{tabular} \\

Valeur maximale = -1, -1, -1, 1, 1, 1, 1, -1 \\
Valeur maximale res  =  6, 2, -2, -6, -6, -2, 2, 6
Changement de signe pour une p�riodicit� de 8 : 2
\end{paragraph}

\begin{paragraph}  {Filtre S5: -1, 0, 2, 0, -1}~ \\
Valeur maximale = 1,1,-1,-1 \\
Valeur maximale res  =  4,4,-4,-4
 Changement de signe pour une p�riodicit� de 8 : 4
\end{paragraph}

\begin{paragraph} {Filtre W5: -1, 2, 0, -2, 1}~ \\
Valeur maximale = -1, -1, 1, -1, 1, 1, -1, 1\\
Valeur maximale res  =  6, 2, 2, 6, 6, 2, 2, 6 (les signes sont faux)
Changement de signe pour une p�riodicit� de 8 : 6
\end{paragraph}

\begin{paragraph} {Filtre R5: 1, -4, 6, -4, 1}~ \\
Valeur maximale = -1, 1 \\
Valeur maximale res  =  16, -16, 16, -16, ......
Changement de signe pour une p�riodicit� de 8 : 8
\end{paragraph}

\subsection{Indices de texture}

\subsection{Calculer le coefficient de pond�ration qui permet de garantir que l'indice de texture prend toujours une valeur comprise entre 0 et 1}

\begin{paragraph} {Filtre L5: 1, 4, 6, 4, 1}~ \\
pour 4 pixel:
En prenant la somme des valeurs absolu on peut avoir une valeur max = 64 \\
Valeur maximale res = 16, 16, 16, 16, 16, 16
res = 1/64
\end{paragraph}

\begin{paragraph} {Filtre E5: -1, -2, 0, 2, 1}~ \\
res = 1/16
\end{paragraph}

\begin{paragraph} {Filtre R5: 1, -4, 6, -4, 1}~ \\
res = 1/16
\end{paragraph}

\begin{paragraph}  {Filtre S5: -1, 0, 2, 0, -1}~ \\
res = 1/16
\end{paragraph}

\begin{paragraph} {Filtre W5: -1, 2, 0, -2, 1}~ \\
res = 1/64
\end{paragraph}


\subsection{Calculer les indices de texture obtenus pour la ligne de pixels de la figure 2 pour le filtre L5 normalis� et R5 normalis�}
















\end{document}