\documentclass[a4paper, 10pt, twocolumn]{article}
\input{/Users/benjamin/Documents/Education/LaTeX/macro.tex}

\title{RDF - S�ance 4 : Les filtres de Laws}
\author{Benjamin \bsc{Van Ryseghem}}

\begin{document}
\maketitle

\section{Filtres de Laws}
\subsection{Question 1}
\subsubsection{Image noire}

\paragraph{Filtre 1}
On obtient que des $-16$.

\paragraph{Filtre 2}
On obtient que des $0$.

\paragraph{Filtre 3}
On obtient que des $0$.

\paragraph{Filtre 4}
On obtient que des $0$.

\paragraph{Filtre 5}
On obtient que des $0$.

\subsubsection{Image blanche}

\paragraph{Filtre 1}
On obtient que des $16$.

\paragraph{Filtre 2}
On obtient que des $0$.

\paragraph{Filtre 3}
On obtient que des $0$.

\paragraph{Filtre 4}
On obtient que des $0$.

\paragraph{Filtre 5}
On obtient que des $0$.

\subsection{Question 2}

\paragraph{Filtre 1}
On obtient que des $0$.

\paragraph{Filtre 2}
On obtient que des $0$.

\paragraph{Filtre 3}
On obtient que des $0$.

\paragraph{Filtre 4}
On obtient que des $0$.

\paragraph{Filtre 5}
On obtient que des $[-16,16,-16,16, \dots\ ]$.


\subsection{Question 3}

\paragraph{Filtre 1}
Il y a $0$ alternance.

\paragraph{Filtre 2}
Il y a $1$ alternance.

\paragraph{Filtre 3}
Il y a $2$ alternance..

\paragraph{Filtre 4}
Il y a $3$ alternance.

\paragraph{Filtre 5}
Il y a $4$ alternance.

\subsection{Question 4}

\paragraph{Filtre 1}
Il faut une ligne constante de $1$.

\paragraph{Filtre 2}
\dots,$-1$,$-1$,$-1$,$-1$,$1$,$1$,$1$,$1$, \dots

On obtient alors \dots, $6$, $2$, $-2$, $-6$, $-6$,$-2$,$2$,$6$, \dots

\paragraph{Filtre 3}
\dots, $-1$, $-1$, $1$, $1$, \dots

On obtient alors \dots, $-4$, $-4$, $4$, $4$, \dots

\paragraph{Filtre 4}
\dots , -1, -1,1,-1,1,1,-1,1 \dots

On obtient alors \dots, $6$, $-6$, $2$, $2$, $-6$, $6$, $-2$, $-2$ \dots

\paragraph{Filtre 5}
Il faut une alternance noir, blanc.

\onecolumn
\signature

\end{document}