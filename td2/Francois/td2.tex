 \documentclass[a4paper,10pt]{article}
\input{/Users/WannaGetHigh/workspace/latex/macros.tex}

\title{TD 2: Le codage des contours}
\author{Fran�ois \bsc{Lepan}}

\begin{document}
\maketitle

\section{R�duction d'une chaine de Freeman}

\subsection{Chaine de Freeman}

\begin{paragraph}{Image 1 :} 31000775754354433 \end{paragraph}
\begin{paragraph}{Image 2 :} 00000767654444213432  \end{paragraph}

\subsection{R�duction chaine de Freeman}

\begin{paragraph}{Annulation:} (0, 4)  (1, 5)  (2, 6)  (3, 7) \end{paragraph}

\begin{paragraph}{Image 1} : 55 \\
La chaine r�sultante correspond aux transformations reliant l'origine et l'arriv�
 \end{paragraph}

\begin{paragraph}{Image 2} (vide) \\
La chaine r�sultante est vide car ferm�e
\end{paragraph}

\subsection{Compl�ter la cha�ne de Freeman}
Il faut "inverser" la chaine r�sultante

\section{Transform�e de Hough d'une image binaire}
\subsection{}
\begin{tabular}{|c|c|c|c|}
\hline
&$-3.5\sqrt[]{2}$&& \\
$0$&&& \\
\hline
$\pi/2$&&& \\
\hline
$\pi/4$&0&& \\
\hline
$3\pi/4$&&& \\
\hline
\end{tabular}
\subsection{}
\end{document}