\documentclass[a4paper,10pt]{article}
\input{/Users/benjamin/Documents/Education/LaTeX/macro.tex}

\title{RDF - S�ance 2}
\author{Benjamin \bsc{Van Ryseghem}}

\begin{document}
\maketitle
\section{Exercice 1 : R�duction d'une chaine de Freeman}
\subsection{Question 1}

\paragraph{Contour 1}
\verb?"3100077575435443"?

\paragraph{Contour 2}
\verb?"00000767654444213432"?

\subsection{Question 2: R�duction de chaine}

On �limine les couples:
\begin{itemize}
	\item (0,4)
	\item (1,5)
	\item (2,6)
	\item (3,7)
\end{itemize}

\paragraph{Contour 1}
\verb?"55"?

La chaine r�sultante correspond aux transformations reliant l'origine et l'arriv�e.


\paragraph{Contour 2}
\verb?""?

La chaine est vide car le contour est ferm�.

\subsection{Question 3}

Il faut "inverser" la chaine r�sultante.

"55" $\rightarrow$ "11"

\section{Exercice 2 : Transform�e de Hough}
\subsection{Question 1}
\subsection{Question 2}
On d�finit le centre du rep�re en (3.5, 3.5)

On obtient alors: \\
\\
\footnotesize
\begin{tabular}{| c | c c c c c c c c c c c c c c |}

\hline
 & $-3.5\sqrt{2}$& $-3\sqrt{2}$&$-2.5\sqrt{2}$&$-2\sqrt{2}$&$-1.5\sqrt{2}$&$-\sqrt{2}$&$-0.5\sqrt{2}$& $0$ & $0.5\sqrt{2}$ & $\sqrt{2}$ & $1.5\sqrt{2}$ & $2\sqrt{2}$ & $2.5\sqrt{2}$ & $3\sqrt{2}$  \\
$\pi/4$ & 0 & x & x & x & 3 & x & x & x &x&x&x&x&x& x \\
$3\pi/4$ & 0 &x &x&x&x&x&x& x & x & x & x & x & x & x \\
\hline
\end{tabular}
\normalsize


\begin{tabular}{| c | c c c c c c c  c |}
\hline
& -3.5 & -2.5 & -1.5 & -0.5 & 0.5 & 1.5 & 2.5 & 3.5 \\
$0$  x	 & x & x & x & x & x & x & x & x \\
$\pi/2$   x	 & x & x & x & x & x & x & x  & x \\
\hline
\end{tabular}

\signature

\end{document}