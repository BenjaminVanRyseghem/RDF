 \documentclass[a4paper,10pt]{article}
\input{/Users/WannaGetHigh/workspace/latex/macros.tex}

\title{RdF : TD sur les textures}
\author{Fran�ois \bsc{Lepan}}

\begin{document}
\maketitle

\section{Matrices de co-occurrence}
\subsection{Calculer les histogrammes des ces trois images et expliquer pourquoi cette mesure statistique ne peut pas �tre exploit�e pour d�crire une texture}

on obtiendra 3 histogramme identique a 20 / 20 /20 / 20 pour chaque niveau de gris

\subsection{calculer les matrices de co-occurrence des niveaux de gris pour tous les couples de d�calages qui permettent de d�crire les paires de pixels voisins en connexit� 8}
M(1;0) M(-1;0)
M(1;-1) M(-1;1)
M(0;-1) M(0;1)
M(-1;-1) M(1;1)

\subsection{Pour la premi�re image de texture, calculer les matrices de co-occurrence pour les couples de d�calages (0,1) et (0,?1)}

???????????????????
\begin{paragraph}{Pour M(0;1) -> pixel du dessous}
\begin{tabular}{|c||c|c|c|c|}
\hline
& 0 & 1 & 2 & 3 \\
\hline
\hline
0& 0 & 0 & 0 &15 \\
\hline
1& 0 & 0 & 20& 0 \\
\hline
2 & 0 &15 & 0 & 0 \\
\hline
3& 20 & 0 & 0 & 0\\
\hline
\end{tabular}
\end{paragraph}
???????????????????

\begin{paragraph}{Pour M(0;-1) -> pixel du dessus}
\begin{tabular}{|c||c|c|c|c|}
\hline
& 0 & 1 & 2 & 3 \\
\hline
\hline
0& 0 & 0 & 0 &15 \\
\hline
1& 0 & 0 & 20& 0 \\
\hline
2 & 0 &15 & 0 & 0 \\
\hline
3& 20 & 0 & 0 & 0\\
\hline
\end{tabular}
\end{paragraph}


\subsection{indiquer pourquoi il n'est pas n�cessaire de calculer les matrices pour tous les couples de d�calage d�termin�s pr�c�demment}



\end{document}